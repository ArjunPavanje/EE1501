
\documentclass{article}
\usepackage{graphicx}
\usepackage{amsmath}
\usepackage{booktabs}
\usepackage{array}
\usepackage{hyperref}
\usepackage{float}
\usepackage{tikz}
\usepackage{circuitikz}
\usepackage{karnaugh-map}
\usepackage{subcaption}
\usepackage{listings}
\usepackage{xcolor}

\title{EE1501 Assignment 1}
\author{Arjun Pavanje}
\date{April 2025}

\begin{document}
\definecolor{codegreen}{rgb}{0,0.6,0}
\definecolor{codegray}{rgb}{0.5,0.5,0.5}
\definecolor{codepurple}{rgb}{0.58,0,0.82}
\definecolor{backcolour}{rgb}{0.95,0.95,0.92}
\lstdefinestyle{style}{
    backgroundcolor=\color{backcolour},   
    commentstyle=\color{codegreen},
    keywordstyle=\color{magenta},
    numberstyle=\tiny\color{codegray},
    stringstyle=\color{codepurple},
    basicstyle=\ttfamily\footnotesize,
    breakatwhitespace=false,         
    breaklines=true,                 
    captionpos=b,                    
    keepspaces=true,                 
    numbersep=5pt,                  
    showspaces=false,                
    showstringspaces=false,
    showtabs=false,                  
    tabsize=2,
    literate={~}{\texttt{\char126}}1,
}
\maketitle
\section{4-to-2 Priority Encoder}
\begin{itemize}
    \item \textbf{Inputs:} A 4-bit input signal \texttt{in[3:0]}.
    \item \textbf{Outputs:}
    \begin{itemize}
        \item \texttt{out[1:0]}: A 2-bit binary output indicating the position of the highest-priority input that is high.
        \item \texttt{valid}: A 1-bit output signal set to 1 if any input bit is high; otherwise, it is set to 0.
    \end{itemize}
\end{itemize}
The priority order of the inputs is: $\texttt{in[3]} > \texttt{in[2]} > \texttt{in[1]} > \texttt{in[0]}$.\newline 
\subsection*{Approach} 
The logic is implemented using a \texttt{casez} statement which allows for the use of don't-care conditions (denoted by \texttt{z} in the input).
\begin{itemize}
    \item When \texttt{in[3]} is high (\texttt{4'b1zzz}), the output \texttt{out} is set to \texttt{3} and \texttt{valid} is set to 1.
    \item When \texttt{in[2]} is high (\texttt{4'b01zz}), the output \texttt{out} is set to \texttt{2} and \texttt{valid} is set to 1.
    \item When \texttt{in[1]} is high (\texttt{4'b001z}), the output \texttt{out} is set to \texttt{1} and \texttt{valid} is set to 1.
    \item When \texttt{in[0]} is high (\texttt{4'b0001}), the output \texttt{out} is set to \texttt{0} and \texttt{valid} is set to 1.
    \item In the default case (when no input is high), the output \texttt{out} is set to \texttt{0} and \texttt{valid} is set to 0.
\end{itemize}

\texttt{casez} saves us from writing out all the cases by letting use don't cares. This way we can intelligently specify \texttt{out, valid} for a few select cases and it works for the entire range of values.
\subsection*{Code}
\begin{lstlisting}[style=style]
module top_module (
    input [3:0] in,
    output reg [1:0] out, 
    output reg[0:0] valid);
    always @(*) begin
        casez (in)
            4'b1zzz: begin 
                out = 2'd3;
                valid = 1;
            end
            4'b01zz: begin
                out = 2'd2; 
                valid = 1;
            end
            4'b001z: begin
                out = 2'd1;
                valid = 1;
            end
            4'b0001: begin
                out = 2'd0;
                valid = 1;
            end
            default: begin
                out = 2'd0;
                valid = 0;
            end
        endcase
    end
endmodule
\end{lstlisting}
\section{4-bit Up Counter (Synchronous)}
\begin{itemize}
    \item \textbf{Inputs:}
    \begin{itemize}
        \item \texttt{clk}: Clock signal.
        \item \texttt{reset}: Asynchronous reset signal (active high).
        \item \texttt{enable}: Control signal to enable counting.
    \end{itemize}
    \item \textbf{Output:}
    \begin{itemize}
        \item \texttt{count[3:0]}: A 4-bit output representing the current count value.
    \end{itemize}
\end{itemize}

\subsection*{Approach}
\begin{itemize}
    \item The counter is initialized to zero at the beginning using the \texttt{initial} block.
    \item The counting behavior is controlled by the \texttt{enable, clock} signal, which allows the counter to increment when high.
    \item When \texttt{reset} is high, the counter is asynchronously reset to zero.
    \item On each positive edge of the \texttt{clk}, \texttt{enable} is checked. If it is high, the counter value increments using Karnaugh-Maps logic (T-flipflops used)
\end{itemize}
\subsection*{State Table}

\begin{center}
\begin{tabular}{|c|c|c|c|c|c|c|c|c|c|c|c|}
\hline
\multicolumn{4}{|c|}{\textbf{Present State}} & \multicolumn{4}{c|}{\textbf{T}} & \multicolumn{4}{c|}{\textbf{Next State}} \\
\hline
$Q_3$ & $Q_2$ & $Q_1$ & $Q_0$ & $T_3$ & $T_2$ & $T_1$ & $T_0$ & $Q_3'$ & $Q_2'$ & $Q_1'$ & $Q_0'$ \\
\hline
0 & 0 & 0 & 0 & 0 & 0 & 0 & 1 & 0 & 0 & 0 & 1 \\
0 & 0 & 0 & 1 & 0 & 0 & 1 & 1 & 0 & 0 & 1 & 0 \\
0 & 0 & 1 & 0 & 0 & 0 & 0 & 1 & 0 & 0 & 1 & 1 \\
0 & 0 & 1 & 1 & 0 & 1 & 1 & 1 & 0 & 1 & 0 & 0 \\
0 & 1 & 0 & 0 & 0 & 0 & 0 & 1 & 0 & 1 & 0 & 1 \\
0 & 1 & 0 & 1 & 0 & 0 & 1 & 1 & 0 & 1 & 1 & 0 \\
0 & 1 & 1 & 0 & 0 & 0 & 0 & 1 & 0 & 1 & 1 & 1 \\
0 & 1 & 1 & 1 & 1 & 1 & 1 & 1 & 1 & 0 & 0 & 0 \\
1 & 0 & 0 & 0 & 0 & 0 & 0 & 1 & 1 & 0 & 0 & 1 \\
1 & 0 & 0 & 1 & 0 & 0 & 1 & 1 & 1 & 0 & 1 & 0 \\
1 & 0 & 1 & 0 & 0 & 0 & 0 & 1 & 1 & 0 & 1 & 1 \\
1 & 0 & 1 & 1 & 0 & 1 & 1 & 1 & 1 & 1 & 0 & 0 \\
1 & 1 & 0 & 0 & 0 & 0 & 0 & 1 & 1 & 1 & 0 & 1 \\
1 & 1 & 0 & 1 & 0 & 0 & 1 & 1 & 1 & 1 & 1 & 0 \\
1 & 1 & 1 & 0 & 0 & 0 & 0 & 1 & 1 & 1 & 1 & 1 \\
1 & 1 & 1 & 1 & 1 & 1 & 1 & 1 & 0 & 0 & 0 & 0 \\
\hline
\end{tabular}
\end{center}

\subsection*{Karnaugh Map for $T_3$}
\begin{karnaugh-map}[4][4][1][$Q_1Q_0$][$Q_3Q_2$]
\manualterms{0,0,0,0,0,0,0,1,0,0,0,0,0,0,0,1}
\implicant{7}{15}
\end{karnaugh-map}
$T_3 = Q_2Q_1Q_0$
\subsection*{Karnaugh Map for $T_2$}
\begin{karnaugh-map}[4][4][1][$Q_1Q_0$][$Q_3Q_2$]
\manualterms{0,0,0,1,0,0,0,1,0,0,0,1,0,0,0,1}
\implicant{3}{11}
\end{karnaugh-map}
$T_2 = Q_1Q_0$
\subsection*{Karnaugh Map for $T_1$}
\begin{karnaugh-map}[4][4][1][$Q_1Q_0$][$Q_3Q_2$]
\manualterms{0,1,0,1,0,1,0,1,0,1,0,1,0,1,0,1}
\implicant{1}{11}
\end{karnaugh-map}
$T_1 = Q_0$

\subsection*{Karnaugh Map for $T_0$}
\begin{karnaugh-map}[4][4][1][$Q_1Q_0$][$Q_3Q_2$]
\manualterms{1,1,1,1,1,1,1,1,1,1,1,1,1,1,1,1}
\implicant{0}{10}
\end{karnaugh-map}
$T_0 = 1$
\subsection*{Code}
\begin{lstlisting}[style=style]
module top_module(
  input clk, 
  input reset,
  input enable,
  output reg[3:0] count
);
    initial begin
    count = 3'd0;
    end
    always @(posedge clk or posedge reset) begin
        if (reset) 
            count = 3'd0;
        else if (enable) begin
            count[0] <= ~count[0];
            count[1] <= count[1] ^ count[0];
            count[2] <= count[2] ^ (count[0] & count[1]);
            count[3] <= count[3] ^ (count[0] & count[1] & count[2]);
        end
    end
endmodule
\end{lstlisting}
\section{Even Parity Generator}
\begin{itemize}
    \item \textbf{Input:}
    \begin{itemize}
        \item \texttt{data[7:0]}: An 8-bit input vector representing the data for which parity is to be calculated.
    \end{itemize}
    \item \textbf{Output:}
    \begin{itemize}
        \item \texttt{parity}: A single-bit output representing the even parity bit.
    \end{itemize}
\end{itemize}
\subsection*{Approach}
The even parity bit is calculated by first performing a bitwise XOR operation across all the bits of the input \texttt{data}. This ensures that parity is 0 when number of 1's is even in the number.\newline

\[
\texttt{parity = (\^{} data)}
\]
generates the even parity bit for the 8-bit input $\texttt{data}$.
\subsection*{Code}
\begin{lstlisting}[style=style]
module top_module(
  input [7:0] data,
  output parity
);
  assign parity = (^data);
endmodule
\end{lstlisting}
\end{document}

